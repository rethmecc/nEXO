about amplitude vs time.

lyod designa function called pulsefinding.exe to find the next pulse. 
We recorded 15000 waveform. each wavefrom has two axis : y : amplitude and x time. 
So we count the time of the next pulse from the first one 
scan the wave form, calculate the standard deviation of the baseline at the standard deviation is a threshold. 
the low bandey /threhsold 4.5 baseline. 

find all local maximum above the baseline

check the certain nb of bin before and after the local maximum,have to the time (1ns fater and 6 ns befoe) and check the amplitude

6 ns : time scale 
1 ns: after help on the right is for after pulse

from the maximeum, check other maxia autour and check the time between difference peak. 
then divide the amplitude of each diff peak by the amplitude of the maxima eq:
ampl(other)/ampl(ampl)/log(distance between two peak)

check the distance and the high of each peak around the maxima. 
very far, and very samll -> not after pulse. 

to avoid oscillating noise, create a thresold obave the baseline and check twice if the oscillating noise cross the neg threshold below the bseline. 

for integrating : pulse cross twice the threhsold and integrate all, no matter aterulse


record the time of the peak, 
record the relative amplitude (for after pulse) and the time

give small slope, because of the quenching resistor,

after 10us it is dn and DN + CT

p[0] = pro of having no after pulse, 
p[1] = DN rate L


