
\documentclass[a4paper, 11pt]{report}

\usepackage{tikz}
\usetikzlibrary{arrows,shapes,trees,backgrounds,positioning,shadows} % loads some tikz extensions
\usepackage{gensymb} % for degree symbol
\usepackage{color}%for color on text

\begin{document}

 %start lower left corner of canvas% start the page 
 %\vspace*{0cm}
 %\hspace{0cm}
 
  As no single university could provide research needs, the initial consertium grows up to 19 different 
  member and associate universities from across Canada. This consortium rules BAlalal and has allowed it to evolve into a national 
  laboratory. %while still maintaining strong ties to the research programms of the CAnadian universities. 
  
  \leavevmode
  \\
  
  \begin{figure}
  \centering 
  \begin{tikzpicture}[scale = 1]%, show background rectangle]
    \draw(0,0) rectangle (11,5);%(x,y)
    %oder has importance 
    
    %pulse shapes and AP
    \draw[-, black, line width=1pt] (0,1)--(3,1);
    \draw[-, black, line width=1pt] (3,1)--(3.8,4);
    \draw[-, black, line width=1pt] (3.8,4)--(5,3);
    \draw[-, black, line width=1pt] (5,3)--(5.3,3.5);
    \draw[-, black, line width=1pt] (5.3,3.5)--(7.3,1);
    \draw[-, black, line width=1pt] (7.3,1)--(11,1);
    
    %script
    \node at(10,0.3) {\color{black} $time (ns)$};
    \node at(1.5,4.6) {\color{black} $Amplitude (mV)$};
    \node at(2.5,2.5) {\color{red} $Amp_{max}$};
    \node at(5.7,3.8) {\color{red} $Ampl_{ap}$};
    \node at(2.9,0.3) {\color{blue} $Time_{max}$};
    \node at(6.1,0.3) {\color{blue} $Time_{ap}$};
    \node at(8,4.5) {\color{black} $R = \frac{Ampl_{ap}}{Amp_{max}}\cdot\ln{(Time_{max}-Time_{ap})}$};
    \node at(9.5,1.7) {\color{green} $Threshold$};
    
    %threshold     
    \draw[-, green, line width=1.3pt, dashed] (0,1.45)--(11,1.45);
    
    % amplitude 
    \draw[|, red, line width=1.3pt] (3.8, 1.45)--(3.8,4);
    \draw[|, red, line width=1.3pt] (5,3)--(5,3.5);
    \draw[-, red, line width=1.3pt,dashed] (5,3.5)--(5.3,3.5);
    
    %time
    \draw[|, red, line width=1.3pt, dashed] (3.8, 0)--(3.8,4);
    \draw[|, red, line width=1.3pt, dashed] (5.3,0)--(5.3,3.5);
    
    
    
  \end{tikzpicture}    
  \caption{The shematic algorithm to find after pulses.Pulse shapes are positive.}
  \label{fig:effect_temp}
  \end{figure}
  
  \begin{figure}
  \centering 
  \begin{tikzpicture}[scale = 1]%, show background rectangle]
    \draw(0,0) rectangle (11,5);%(x,y)
    %oder has importance 
    
    \draw[|, black!80, line width=1pt] (6.55,1.2)--(6.55,2.5);
    \draw[-, black!80, line width=1pt] (6.54,2.5)--(7.25,2.5);
    \draw[|, black!80, line width=1pt] (7,2.8)--(7,2.5);
    \draw[|, black!80, line width=1pt] (7.25,2.88)--(7.25,0.88);
    \draw[-, black!80, line width=1pt] (0,0.88)--(11,0.88) node at (6,0.4) {\color{black} \textbf{$Bottom\ of\ the\ box$}};
    
    \draw[-,black,thick,dashed] (0,3.35) node[above right]{\color{black} $Beam \ @ -100^{\circ}$} -- (7.2,3.35) ;%cooling    
    \draw[-,red,thick,dashed] (0,3) node[below right]{\color{black} $Beam \ @ \ RT$} -- (7.2,3);%RT
    \draw[|,blue,line width=1.5pt] (7,4) -- (7,3.7);%collimator top top
    \draw[|,blue,line width=1.5pt] (7,2.8) -- (7,3.1);%collimator top bottom     
    \draw[black](7.2,2.9) rectangle (7.3,3.8); % top detector
    \draw (9,1.8) node at (9,2) {$Photodetectors$};
    \draw[->] (8,2.3) -- (7.4,3.35);%arrows to top
    \draw[->] (8,1.7) -- (6.6,1);% arrow to bottom
    
    \draw[-,green,thick,dashed] (5.66,3.35) -- (5.66,1);%cooling    
    \draw[-,red,thick,dashed] (6,3) -- (6,1);%RT
    \draw[-,blue,line width=1.5pt] (5.46,1.2) -- (5.76,1.2);%collimator top right
    \draw[-,blue,line width=1.5pt] (6.25,1.2) node at(4,1.2) {\color{blue} $Collimator$} -- (6.55,1.2);%collimator top left
    \draw[black](5.56,0.9) rectangle (6.45,1);% bottom detector 
    
    \draw[|, black!80, line width=1pt] (6.55,1.2)--(6.55,2.5);
    \draw[-, black!80, line width=1pt] (6.54,2.5)--(7.25,2.5);
    \draw[|, black!80, line width=1pt] (7,2.8)--(7,2.5);
    \draw[|, black!80, line width=1pt] (7.25,2.88)--(7.25,0.88);
    \draw[-, black!80, line width=1pt] (0,0.88)--(11,0.88) node at (6,0.4) {\color{black} \textbf{$Bottom\ of\ the\ box$}};
    
    
    \draw[-,black, thick] (5.2,3.8) -- +(315:1.9) node at(4.8,4.3) {$Beam splitter$};%bs
    
  \end{tikzpicture}    
  \caption{Effect of the temperature on the position of the two photodetectors}
  \label{fig:effect_temp}
  \end{figure}
  
  
  \begin{figure}[!hbtp]
    \centering
    \frame{\includegraphics[totalheight=0.2\textwidth,trim=0cm 0cm 0cm 0cm, clip=true,page= 2]{../Characterization_of_VUV_sensitive_SiPMs_for_nEXO.pdf}}
    \caption{Dark and light regions.}
    \label{fig:dark_light_region}
  \end{figure}
  
  
\end{document}
